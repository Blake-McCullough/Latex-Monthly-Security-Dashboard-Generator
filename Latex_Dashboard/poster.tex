
%Made By Blake McCullough❤
%Discord - Spoiled_Kitten#4911
%Github - https://github.com/Blake-McCullough/
%Email - privblakemccullough@protonmail.com

%%%% NOTE FROM LEARNING, DO NOT SET ANYTHING TO VARIABLE \COUNT!!!

%----------------------------------------------------------------------------------------
%    PAGE TYPE DECLARING
%----------------------------------------------------------------------------------------
\documentclass[a0paper,landscape]{baposter}
%Colour mode style
\selectcolormodel{cmyk}

%----------------------------------------------------------------------------------------
%    DECLARES THE PACKAGES NEEDED
%----------------------------------------------------------------------------------------
\usepackage{wrapfig}
\usepackage{lmodern}
\usepackage{lipsum,graphicx}
\usepackage[utf8]{inputenc} %unicode support
\usepackage[T1]{fontenc}
%PACKAGE FOR MONTH AND DATE
\usepackage{datetime}
% PIE CHART DRAWING LIBRARY
\usepackage{pgf-pie}
%CSV FILE READING LIBRARY
\usepackage{datatool}
%For generating bar graph
\usepackage{bchart}
%For trend line graph
\usepackage{pgfplots}
\usepackage{pgfplotstable}
%setting version
\pgfplotsset{compat=1.17}
%For graphs
\usepackage{tikz}
%USED TO COMPARE STRINGS
\usepackage{xstring}


%----------------------------------------------------------------------------------------
%    CSV FILE SETUP
%----------------------------------------------------------------------------------------
%SETS UP CSV FILE FOR PICKING UP VARIABLES FOR MAIN USE
\DTLsetseparator{,}
\DTLloaddb{database}{database.csv}

%SETS UP CSV FILE FOR PICKING UP VARIABLES FOR PAST SIX MONTHS
\DTLsetseparator{,}
\DTLloaddb{pastsixmonths}{pastsixmonths.csv}



%----------------------------------------------------------------------------------------
%    Calculates for previous months
%----------------------------------------------------------------------------------------

% \setdate{<day>}{<month>}{<year>}
\newcommand{\setdate}[3]{\day=#1\month=#2\year=#3\relax}%




%Calculates the last month
\newcommand{\lastmonth}{%
  \ifcase\month\or%  0
    December\or   %  1
    January\or    %  2
    February\or   %  3
    March\or      %  4
    April\or      %  5
    May\or        %  6
    June\or       %  7
    July\or       %  8
    August\or     %  9
    September\or  % 10
    October\or    % 11
    November\fi   % 12
}


%Calculates 6 months ago month

\newcommand{\sixmonthprevious}{%
  \ifcase\month\or%  0
    July\or   %  1
    August\or    %  2
    September\or   %  3
    October\or      %  4
    November\or      %  5
    December\or        %  6
    January\or       %  7
    February\or       %  8
    March\or     %  9
    April\or  % 10
    May\or    % 11
    June\fi   % 12
}

%----------------------------------------------------------------------------------------
%    Random stuff
%----------------------------------------------------------------------------------------
\newcommand{\compresslist}{%
\setlength{\itemsep}{0pt}%
\setlength{\parskip}{1pt}%
\setlength{\parsep}{0pt}%
}

\newenvironment{boenumerate}
  {\begin{enumerate}\renewcommand\labelenumi{\textbf\theenumi.}}
  {\end{enumerate}}


\begin{document}
%----------------------------------------------------------------------------------------
%    setting colour code
%----------------------------------------------------------------------------------------
\definecolor{Mycolor1}{HTML}{00FFFF}
\definecolor{Mycolor2}{HTML}{00AEEF}

\begin{poster}
{
%----------------------------------------------------------------------------------------
%    SETUP FOR PAGE
%----------------------------------------------------------------------------------------
grid=false,
headerborder=open, % Adds a border around the header of content boxes
colspacing=1em, % Column spacing
bgColorOne=white, % Background color for the gradient on the left side of the poster
bgColorTwo=white, % Background color for the gradient on the right side of the poster
borderColor=Mycolor1, % Border color
headerColorOne=Mycolor2, % Background color for the header in the content boxes (left side)
headerColorTwo=Mycolor2, % Background color for the header in the content boxes (right side)
headerFontColor=white, % Text color for the header text in the content boxes
boxColorOne=white, % Background color of the content boxes
textborder=rounded, %rectangle, % Format of the border around content boxes, can be: none, bars, coils, triangles, rectangle, rounded, roundedsmall, roundedright or faded
eyecatcher=false, % Set to false for ignoring the left logo in the title and move the title left
headerheight=0.11\textheight, % Height of the header
headershape=rounded, % Specify the rounded corner in the content box headers, can be: rectangle, small-rounded, roundedright, roundedleft or rounded
headershade=plain,
headerfont=\Large\textsf, % Large, bold and sans serif font in the headers of content boxes
%textfont={\setlength{\parindent}{1.5em}}, % Uncomment for paragraph indentation
linewidth=2pt % Width of the border lines around content boxes
}{}
%----------------------------------------------------------------------------------------
%	TITLE AND AUTHOR NAME
%----------------------------------------------------------------------------------------
{
\textsf %Sans Serif
{\vspace{0.3em}\\
{\input{CompanyName.txt}}
}
} %Company title
% {\small \vspace{0.7em} Department of Computing, TU Dublin, Tallaght, Dublin, Ireland}} 
{\sf\vspace{0.2em}\\
Security Review for \lastmonth % Sets the Month for the current review period
\vspace{0.3em}\\
\small{ Created by Blake McCullough's Software %KEEP AS IS CREDIT
\vspace{0.2em}\\

}
}
{\includegraphics[width=.13\linewidth]{company-logo.png}} % logo


%----------------------------------------------------------------------------------------
%    pulling data from CSV
%----------------------------------------------------------------------------------------
%Total database from 6 months prior
\DTLassignfirstmatch{database}{count}{totalattackssixmonth}{\counted=count,\totalattacksixmonth=points}
%Total attacks of current month
\DTLassignfirstmatch{database}{count}{totalattacks}{\counted=count,\totalattackcurrentmonth=points}
%Gets the value from six months ago for the top attack
\DTLassignfirstmatch{database}{count}{commonattack}{\counted=count,\sixmonthsagoonecommonpoints=points,\sixmonthsagoonecommon=name}
%One month ago
\DTLassignfirstmatch{pastsixmonths}{month}{1}{\month=month,\onemonthpoints=points,\onemonth=name}
%two months ago
\DTLassignfirstmatch{pastsixmonths}{month}{2}{\month=month,\twomonthpoints=points,\twomonth=name}
%three months ago
\DTLassignfirstmatch{pastsixmonths}{month}{3}{\month=month,\threemonthpoints=points,\threemonth=name}
%four months ago
\DTLassignfirstmatch{pastsixmonths}{month}{4}{\month=month,\fourmonthpoints=points,\fourmonth=name}
%five months ago
\DTLassignfirstmatch{pastsixmonths}{month}{5}{\month=month,\fivemonthpoints=points,\fivemonth=name}
%six months ago
\DTLassignfirstmatch{pastsixmonths}{month}{6}{\month=month,\sixmonthpoints=points,\sixmonth=name}
%First most common attack
\DTLassignfirstmatch{database}{count}{1}{\counted=count,\onepoints=points,\onename=name}
%Second most common attack
\DTLassignfirstmatch{database}{count}{2}{\counted=count,\twopoints=points,\twoname=name}
%Third most common attack
\DTLassignfirstmatch{database}{count}{3}{\counted=count,\threepoints=points,\threename=name}
%Fourth most common attack
\DTLassignfirstmatch{database}{count}{4}{\counted=count,\fourpoints=points,\fourname=name}
%Fifth most common attack
\DTLassignfirstmatch{database}{count}{5}{\counted=count,\fivepoints=points,\fivename=name}
%Sixth most common attack
\DTLassignfirstmatch{database}{count}{6}{\counted=count,\sixpoints=points,\sixname=name}
%Seventh most common attack
\DTLassignfirstmatch{database}{count}{7}{\counted=count,\sevenpoints=points,\sevenname=name}
%Eighth most common attack
\DTLassignfirstmatch{database}{count}{8}{\counted=count,\eightpoints=points,\eightname=name}
%Ninth most common attack
\DTLassignfirstmatch{database}{count}{9}{\counted=count,\ninepoints=points,\ninename=name}
%Tenth most common attack
\DTLassignfirstmatch{database}{count}{10}{\counted=count,\tenpoints=points,\tenname=name}
%High occurrences
\DTLassignfirstmatch{database}{count}{High}{\counted=count,\High=points}
%Medium occurrences
\DTLassignfirstmatch{database}{count}{Medium}{\counted=count,\Medium=points}
%Low occurrences
\DTLassignfirstmatch{database}{count}{Low}{\counted=count,\Low=points}
%Informational Occurrences
\DTLassignfirstmatch{database}{count}{Informational}{\counted=count,\Informational=points}
    
    




% this states the box starts at column 0 (edge of page), directly below the box labelled introduction for a span of 1 (column wide)
\headerbox{Top 10 types of attacks}{name=subtopic1,column=0,row=0,span=1}{
%----------------------------------------------------------------------------------------
%    CREATES PIE GRAPH WITH VARIABLES DECLARED
%----------------------------------------------------------------------------------------
\begin{tikzpicture}

\pie[
    color = {
    %one COLOUR
        yellow!90!black,
    %two COLOUR
        green!60!black,
    %three COLOUR
        blue!60,
    %four COLOUR
        red!70, 
    %five COLOUR
        gray!70,
    %six COLOUR
        teal!20,
    %seven COLOUR
        black!50,
    %eight COLOUR
        green!90,
    %nine COLOUR
        yellow!20,
    %Ten COLOUR
        blue!20}, 
    sum = auto %AUTOMATICALLY WORKS OUT PERCENTAGE THAT EACH NUMBER IS OVERALL
]{\onepoints/\onename, %SETS NUMBER FOR THE ELEMENT FROM VARIABLE
    \twopoints/\twoname, %SETS NUMBER FOR THE ELEMENT FROM VARIABLE
    \threepoints/\threename, %SETS NUMBER FOR THE ELEMENT FROM VARIABLE
    \fourpoints/\fourname, %SETS NUMBER FOR THE ELEMENT FROM VARIABLE
    \fivepoints/\fivename, %SETS NUMBER FOR THE ELEMENT FROM VARIABLE
    \sixpoints/\sixname, %SETS NUMBER FOR THE ELEMENT FROM VARIABLE
    \sevenpoints/\sevenname, %SETS NUMBER FOR THE ELEMENT FROM VARIABLE
    \eightpoints/\eightname, %SETS NUMBER FOR THE ELEMENT FROM VARIABLE
    \ninepoints/\ninename, %SETS NUMBER FOR THE ELEMENT FROM VARIABLE
    \tenpoints/\tenname %SETS NUMBER FOR THE ELEMENT FROM VARIABLE
}
 
\end{tikzpicture}
 


}


% this states the box starts at column 0 (edge of page), directly below the box labelled introduction for a span of 1 (column wide)
\headerbox{Severity of Attempts}{name=subtopic2,column=0,below=subtopic1,span=1}{
 %----------------------------------------------------------------------------------------
%CREATES PIE GRAPH WITH VARIABLES DECLARED
%----------------------------------------------------------------------------------------
\begin{tikzpicture}
     
\pie[
    color = {
    %High COLOUR
        red!90,
    %Medium COLOUR
        yellow!60,
    %Low COLOUR
        green!60,
    %Informational COLOUR
        blue!70
        },
    sum = auto %AUTOMATICALLY WORKS OUT PERCENTAGE THAT EACH NUMBER IS OVERALL
]{\High/High, %SETS NUMBER FOR THE ELEMENT FROM VARIABLE
    \Medium/Medium, %SETS NUMBER FOR THE ELEMENT FROM VARIABLE
    \Low/Low, %SETS NUMBER FOR THE ELEMENT FROM VARIABLE
    \Informational/Informational %SETS NUMBER FOR THE ELEMENT FROM VARIABLE
    }
 
\end{tikzpicture}
 

    

}
% this states the box starts at column 0 (edge of page), directly below the box labelled introduction for a span of 1 (column wide)
\headerbox{Total Attacks compared to 6 months ago}{name=subtopic3,column=1,row=0,span=2}{
\vspace{1cm} %remove this, only added for spacing
 %----------------------------------------------------------------------------------------
 %    Bar chart for the total attacks
 %----------------------------------------------------------------------------------------
 \begin{center} 
%BAR CHART WHICH TAKES VARIABLES FROM CSV AND PUT THEM ON GRAPH, ALSO GRABS THE CORRECT MONTH


   \begin{bchart}[step=20]
        \bigskip
        \bclabel{Total Attacks}
        \bcbar[color=yellow]{\totalattacksixmonth}
        \bclabel{in \sixmonthprevious }
        \bigskip
        \bclabel{Total Attacks}
        \bcbar{\totalattackcurrentmonth}
        \bclabel{in \lastmonth }
        \bigskip
        

    \end{bchart}
 
\end{center} 
\vspace{1cm} %remove this, only added for spacing


}


% this states the box starts at column 0 (edge of page), directly below the box labelled introduction for a span of 1 (column wide)
\headerbox{Sub Topic 4}{name=subtopic4,column=3,row=0,span=1}{
\vspace{0.5cm}
 \begin{center} 
{\includegraphics[width=7cm,height=5cm]{placeholder.png}}
 \end{center} 
\vspace{0.5cm}

}

% this states the box starts at column 0 (edge of page), directly below the box labelled introduction for a span of 1 (column wide)
\headerbox{Attacks over past 6 months}{name=subtopic5,column=3,below=subtopic4,span=1}{
%----------------------------------------------------------------------------------------
%    trend line for past six months
%----------------------------------------------------------------------------------------
\begin{tikzpicture}
%plot 0 is six months ago, plot 1 is five months ago etc
    \begin{axis}
        [
        ,width=8.5cm
        ,xlabel=Test
        ,ylabel=Mean
        ,xtick=data
       %,xtick={0,1,...,3}
        ,xticklabels={\onemonth,\twomonth,\threemonth,\fourmonth,\fivemonth,\sixmonth}
        ]
        \addplot+[sharp plot] coordinates
        {(0,\onemonthpoints) (1,\twomonthpoints) (2,\threemonthpoints) (3,\fourmonthpoints) (4,\fivemonthpoints) (5,\sixmonthpoints)};
    \end{axis}
\end{tikzpicture}

\vspace{0.15cm}





}
% this states the box starts at column 1 (left middle of page), directly below the box labelled introduction for a span of 2 (column wide)
\headerbox{Summary}{name=Summary,column=1,below=subtopic3,span=2}{


\vspace{0.15cm}
%----------------------------------------------------------------------------------------
%    Automatic text filling based on set conditions to compare six months ago
%----------------------------------------------------------------------------------------


The most common attack this month is \onename{} which happened \onepoints{} times.
\IfStrEq{\sixmonthsagoonecommon}{\onename}\\
{This is the same as last time where it occurred \sixmonthsagoonecommonpoints{} times}{ This is Different to last time where it was \sixmonthsagoonecommon{}  that occurred \sixmonthsagoonecommonpoints{} times.}

%----------------------------------------------------------------------------------------
%    Compares the attacks to work if same, less or greater
%----------------------------------------------------------------------------------------

%IF the number of attacks is less it says it has fallen
\ifnum \totalattackcurrentmonth<\totalattacksixmonth
{There has been a fall in the amount of attacks.}

\else
    \ifnum \totalattackcurrentmonth>\totalattacksixmonth
    %if the number is higher than it says it has risen
    {There has been a rise in the amount of attacks.}
    \else
    %If it has stayed the same it says that it doesn't change
    {There has been no change in the amount of attacks.}
    \fi
\fi






\vspace{5.1cm} %remove this, only added for spacing

}


\end{poster}

\end{document}
